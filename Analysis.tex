\documentclass[11pt,]{article}
\usepackage[]{palatino}
\usepackage{amssymb,amsmath}
\usepackage{ifxetex,ifluatex}
\usepackage{fixltx2e} % provides \textsubscript
\ifnum 0\ifxetex 1\fi\ifluatex 1\fi=0 % if pdftex
  \usepackage[T1]{fontenc}
  \usepackage[utf8]{inputenc}
\else % if luatex or xelatex
  \ifxetex
    \usepackage{mathspec}
  \else
    \usepackage{fontspec}
  \fi
  \defaultfontfeatures{Ligatures=TeX,Scale=MatchLowercase}
\fi
% use upquote if available, for straight quotes in verbatim environments
\IfFileExists{upquote.sty}{\usepackage{upquote}}{}
% use microtype if available
\IfFileExists{microtype.sty}{%
\usepackage{microtype}
\UseMicrotypeSet[protrusion]{basicmath} % disable protrusion for tt fonts
}{}
\usepackage[margin=1in]{geometry}
\usepackage{hyperref}
\hypersetup{unicode=true,
            pdftitle={Unit 12 - Homework},
            pdfauthor={\textbar{} Grant Bourzikas \textbar{} Quinton Nixon \textbar{} Brock Friedrich},
            pdfborder={0 0 0},
            breaklinks=true}
\urlstyle{same}  % don't use monospace font for urls
\usepackage{graphicx,grffile}
\makeatletter
\def\maxwidth{\ifdim\Gin@nat@width>\linewidth\linewidth\else\Gin@nat@width\fi}
\def\maxheight{\ifdim\Gin@nat@height>\textheight\textheight\else\Gin@nat@height\fi}
\makeatother
% Scale images if necessary, so that they will not overflow the page
% margins by default, and it is still possible to overwrite the defaults
% using explicit options in \includegraphics[width, height, ...]{}
\setkeys{Gin}{width=\maxwidth,height=\maxheight,keepaspectratio}
\IfFileExists{parskip.sty}{%
\usepackage{parskip}
}{% else
\setlength{\parindent}{0pt}
\setlength{\parskip}{6pt plus 2pt minus 1pt}
}
\setlength{\emergencystretch}{3em}  % prevent overfull lines
\providecommand{\tightlist}{%
  \setlength{\itemsep}{0pt}\setlength{\parskip}{0pt}}
\setcounter{secnumdepth}{0}
% Redefines (sub)paragraphs to behave more like sections
\ifx\paragraph\undefined\else
\let\oldparagraph\paragraph
\renewcommand{\paragraph}[1]{\oldparagraph{#1}\mbox{}}
\fi
\ifx\subparagraph\undefined\else
\let\oldsubparagraph\subparagraph
\renewcommand{\subparagraph}[1]{\oldsubparagraph{#1}\mbox{}}
\fi

%%% Use protect on footnotes to avoid problems with footnotes in titles
\let\rmarkdownfootnote\footnote%
\def\footnote{\protect\rmarkdownfootnote}

%%% Change title format to be more compact
\usepackage{titling}

% Create subtitle command for use in maketitle
\newcommand{\subtitle}[1]{
  \posttitle{
    \begin{center}\large#1\end{center}
    }
}

\setlength{\droptitle}{-2em}

  \title{Unit 12 - Homework}
    \pretitle{\vspace{\droptitle}\centering\huge}
  \posttitle{\par}
    \author{\textbar{} Grant Bourzikas \textbar{} Quinton Nixon \textbar{} Brock
Friedrich}
    \preauthor{\centering\large\emph}
  \postauthor{\par}
      \predate{\centering\large\emph}
  \postdate{\par}
    \date{August 10, 2018}

\usepackage{amsmath}
\usepackage{mathtools}
\usepackage{float}
\usepackage{xcolor,pifont}
\usepackage{ulem}
\newcommand{\cmark}{\Large\textcolor{green}{\ding{52}}}
\newcommand{\xmark}{\Large\textcolor{red}{\ding{55}}}
\definecolor{answercolor}{RGB}{35,155,86}
\newcommand{\answerblock}{\textcolor{answercolor}}

\begin{document}
\maketitle

{
\setcounter{tocdepth}{2}
\tableofcontents
}
\noindent\makebox[\linewidth]{\rule{\textwidth}{0.4pt}}

\section*{Introduction}

Ask a home buyer to describe their dream house, and they probably won't
begin with the height of the basement ceiling or the proximity to an
east-west railroad. However, it is essential to review the data because
it proves that there are many other influences in price negotiations
than the number of bedrooms or a white-picket fence.

\noindent\makebox[\linewidth]{\rule{\textwidth}{0.4pt}}

\section{Data Synopsis}

The Ames House dataset was compiled by Dean De Cock and contains 79
explanatory variables describing almost every aspect of residual home in
Ames Iowa from 2006 to 2010. The data set contains 2930 observations
involved in assessing home values.

\noindent\makebox[\linewidth]{\rule{\textwidth}{0.4pt}}

\section{Analysis Question 1}

\textbackslash{}subsection(Restatement of Problem)

To build and fit a model, an analysis must be performed to identify
features of the dataset that are statistically significant in their
relation to, and prediction of, the sales price.

\subsection{Build and Fit the Model}

\subsubsection{Interrogate the Data}

To build and fit a model, an analysis must be performed to identify
features of the dataset that are statistically significant in their
relation to, and prediction of, the sales price.

\begin{itemize}
\item{ Plot the data.}
\item{ Develop a tentative model(s).}
\item{ Using the question(s) of interest (QOI).}
\item{ Accounting for confounders.}
\item{ Accounting for relationships ($X^2$,$X^3$, $etc$).}
\item{ Fit the model(s).}
\item{ Evaluate residual plots.}
\item{ Constant SD.}
\item{ Normality and zero mean.}
\item{ Identify any influential observations.}
\end{itemize}

• Ames\^{}SalesPrice = B0 + B1\emph{BrkSide + B2}Edwards +
B3\emph{NAmes + B4(LogLivingArea}BrkSide) + B5(LogLivingArea*Edwards)

\(\mu\widehat\{SalesPrice_{Ames}}\,=\,\beta_0 + \beta_1\,*\,BrkSide + \beta_2\,Edwards\,+\,\beta_3\,*\,NAmes + \beta_4(LivingArea_{log}\, x\,BrkSide) + \beta_{5}\,x\,(LivingArea_{log}\,x\,Edwards)\)

• Ames\^{}SalesPrice = 8.49 + (-2.58\emph{BrkSide) + (-0.49}Edwards) +
(0.0 * NAmes) + (B3\emph{0.0) + B4(0.47}BrkSide) + B5(0.47*Edwards)

\noindent\makebox[\linewidth]{\rule{\textwidth}{0.4pt}}

\section{Analysis Question 2}


\end{document}
